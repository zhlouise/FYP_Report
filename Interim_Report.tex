\documentclass[12pt, a4paper]{article}


% --- Packages ---
\usepackage[utf8]{inputenc}
\usepackage[T1]{fontenc}
\usepackage{times} % Sets font to Times New Roman
\usepackage[margin=2.5cm]{geometry} % Sets 2.5cm margins
\usepackage{setspace} % Required for double spacing
\usepackage{titlesec}
\usepackage{sectsty}
\usepackage{cite}
\usepackage{graphicx}
\usepackage{soul}
\usepackage{color}


% --- Settings ---
\doublespacing % Applies double spacing to the document
\sectionfont{\normalsize}
\subsectionfont{\normalsize\it\bfseries}
\subsubsectionfont{\normalsize\it}


\begin{document}


% --- Title Page ---
\begin{titlepage}
    \centering
    \includegraphics[width=0.5\textwidth]{Included_Images/PolyU_Logo.png}
    \hfill
    \includegraphics[width=0.25\textwidth]{Included_Images/FENG_Logo.png}
    \vfill
    {\Large 2025/26 Interdepartmental Final Year Project Interim Report\par}
    {\huge \textbf{Intelligent UAV Systems for GNSS-Based Remote Sensing on
    Vegetation} \par}
    \vfill
    {\large [AAE] ZHOU Jiayi (22099961D) \par} {\large [AAE] MAHMUD Md Sahat
    (22097159D) \par} {\large [EEE] GAMAGE Sashenka (22097129D) \par} {\large
    [ME] TAN Qing Lin (22101126D) \par}
    \vfill
    {\large \textbf{Chief Supervisor: } [AAE] Prof. Guohao ZHANG \par} {\large
    \textbf{Co-Supervisor: } [AAE] Prof. Li-Ta HSU \par}
    \vspace{1cm}
    {\large \textbf{Date of Submission: } 2026 January 18 \par}
\end{titlepage}


% --- Table of Contents ---
\pagenumbering{roman}
\tableofcontents
\newpage


% --- Main Content ---
\pagenumbering{arabic}


% --- Abstract ---
\section{Abstract}
Placeholder text


% --- Introduction ---
\section{Introduction}
Vegetation is a critical asset to the environment and the human civilization.
Not only does vegetation produce essential societal resources, but its
distribution and productivity also greatly impacts the terrestrial ecosystems
and the global climate \cite{Richardson2013}. Therefore, the continuous and
accurate monitoring of vegetation is essential for sustainable resource
management \cite{Wallace2006}, ecosystem preservation \cite{Zeng2022}, and
climate change modeling \cite{Richardson2013}. 

Traditionally, methods of vegetation monitoring involved manual field
assessments of site characteristics, extracting vegetation condition indicators
such as species composition, geometrical structure, and biochemical activities
\cite{Gibbons2006}. However, the accuracy and efficiency of such monitoring
methods are highly dependent on the available expertise and resources, as the
site assessments often required the assessor to possess reasonable levels of
field knowledge prior to surveying \cite{Parkes2003}. To address the limitations
of manual assessments and to accommodate for the increasing large-scale
monitoring demands, remote sensing systems emerged as an essential tool for
ecological monitoring \cite{Li2023}. 

Remote sensing refers to the acquisition of an object's information through
measurements obtained without coming into direct contact with said object,
effectively minimizing the need for manual involvement on-site
\cite{campbell2011introduction}. Specifically, remote sensing relies on
information derived from different measurements of energy reflected from the
object of interest \cite{campbell2011introduction}. In the context of surveying
terrestrial vegetation, the remote sensing methods could be classified into two
categories based on the distance between the target object and the measurement
sensor: 1) space-borne and 2) airborne remote sensing. Space-borne remote
sensing involves the use of instruments onboard orbiting satellites, including
spectral and hyperspectral cameras \cite{Qian2022}, synthetic aperture radar
(SAR) \cite{Hu2025}, and space-borne Light Detection And Ranging (LiDAR) sensors
\cite{Bergen2009}. Due to the wide coverage and availability of space-borne
data, large-scale terrestrial changes over time could be captured, enabling the
continuous monitoring of macro-scale ecosystems \cite{Khan2024}. Compared to
space-borne systems, airborne remote sensing can provide significantly improved
spatial resolution and assessment flexibility through the integration of sensors
onboard manned aircraft or unmanned aerial vehicles (UAVs) \cite{Khan2024}.
Although limited by coverage area, airborne remote sensing can provide timely
information for addressing regional emergencies such as pest \cite{Wulder2006}
or wildfire \cite{Arroyo2008} outspreads since the systems could be deployed
on-demand. In UAV-based remote sensing particularly, this temporal flexibility
is further complemented with the benefit of low operational cost, rendering it
an ideal platform for monitoring regional and urban vegetation \cite{Tang2015}. 

Conventionally, UAV remote sensing platforms carry similar instruments to that
of other remote sensing systems, including radar, LiDAR, and multi-spectral or
hyperspectral imagery sensors \cite{Tang2015}. However, while imaging
instruments are susceptible to the influence of lighting and weather
\cite{Wu2024}, LiDAR devices tend to face difficulties penetrating through dense
canopy \cite{Su2007}. Therefore, it is critical to explore a robust sensing
technique that is suitable for UAVs in terms of payload and power consumption to
complement the existing instruments. Recently, the technique of Global
Navigation Satellite System Reflectometry (GNSS-R) is receiving increasing
interest in the field of remote sensing. GNSS-R exploits the L-band signals
transmitted from Global Navigation Satellite Systems (GNSS) that are then
scattered on different terrain surfaces of the Earth \cite{Jin2024}. Then, the
reception of reflected GNSS signals can provide information regarding the
properties of the signal reflector on land \cite{Jin2024}. As signals of
opportunity conventionally dedicated to Positioning, Navigation, and Timing
(PNT) applications, GNSS is capable of providing real-time measurements
regardless of time, location, and weather \cite{Jin2024}. Additionally, as a
bi-static system where signal transmitters are separated from receivers, GNSS-R
is exempt from the need of dedicated transmitter-receiver instruments that are
crucial to mono-static radar systems \cite{Jin2010}. Instead, any consumer-grade
receivers capable of receiving reflected GNSS signals could be used for GNSS-R
remote sensing, further demonstrating GNSS-R's applicability onboard low-cost
remote sensing systems such as an UAV. 

\hl{May need some intro for path planning and lidar here. The above is from
Louise's project proposal, so need further editing.}

In this project, the technique of GNSS-R will be integrated onto an UAV platform
to achieve an intelligent and structured remote sensing system for vegetation
monitoring. The main objectives of this project are as follows:
\begin{enumerate}
\item To introduce an autonomous path planning framework onboard the UAV platform
for optimized GNSS-R remote sensing.
\item To analyze and model the correlations between signal propagation parameters
retrieved through GNSS-R and ground vegetation conditions.
\item To classify signals reflected by vegetation and predict vegetation parameters
based on raw GNSS data using machine learning-based approach.
\item To establish a detailed 3D canopy map using LiDAR-based SLAM, providing a
spatial validation reference for the 2D vegetation features detected by GNSS-R.
\end{enumerate}

This report is structured as follows: Section 3 reviews contemporary research in
UAV path planning, GNSS-R, machine learning, and LiDAR SLAM, while Section 4
outlines the project's methodologies. The subsequent sections cover the
experiments conducted (Section 5), a discussion of current results (Section 6),
and the project conclusions (Section 7). Finally, Section 8 explores future
work, and Section 9 summarizes project management details. 


% --- Literature Review ---
\section{Literature Review}
Placeholder text

\subsection{Path Planning}
Placeholder text

\subsection{GNSS-R}
Placeholder text

\subsection{Machine Learning}
Placeholder text

\subsection{LiDAR SLAM}
Placeholder text


% --- Methodology ---
\section{Methodology}
Placeholder text

\subsection{UAV Platform}
Placeholder text

\subsection{System Architecture}
Placeholder text

\subsection{Path Planning}
Placeholder text

\subsection{GNSS-R}
Placeholder text

\subsection{Machine Learning}
Placeholder text

\subsection{LiDAR SLAM}
Placeholder text


% --- Experiments ---
\section{Experiments}
Placeholder text

\subsection{Experiment Workflow}
Placeholder text

\subsection{Summary of Experiments Conducted}
Placeholder text

\subsection{Key Experiment 1}
Placeholder text

\subsection{Key Experiment 2}
Placeholder text


% --- Results and Discussion ---
\section{Results and Discussion}
Placeholder text

\subsection{Path Planning}
Placeholder text

\subsection{GNSS-R}
Placeholder text

\subsection{Machine Learning}
Placeholder text

\subsection{LiDAR SLAM}
Placeholder text


% --- Conclusion ---
\section{Conclusion}
Placeholder text


% --- Future Works ---
\section{Future Works}
Placeholder text

\subsection{Path Planning}
Placeholder text

\subsection{GNSS-R}
Placeholder text

\subsection{Machine Learning}
Placeholder text

\subsection{LiDAR SLAM}
Placeholder text


% --- Project Management ---
\section{Project Management}
Placeholder text

\subsection{Gantt Chart}
Placeholder text

\subsection{Project Difficulties and Solutions}
Placeholder text

\subsubsection{Path Planning}
Placeholder text

\subsubsection{GNSS-R}
Placeholder text

\subsubsection{Machine Learning}
Placeholder text

\subsubsection{LiDAR SLAM}
Placeholder text


% --- Appendix ---
\newpage
\addcontentsline{toc}{section}{Appendix}
\section*{Appendix}
Placeholder text


% --- References ---
\newpage
\addcontentsline{toc}{section}{References}
\bibliographystyle{IEEEtran}
\bibliography{References.bib} 


\end{document}