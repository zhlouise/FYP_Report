\documentclass[12pt, a4paper]{article}


% --- Packages ---
\usepackage[utf8]{inputenc}
\usepackage[T1]{fontenc}
\usepackage{times} % Sets font to Times New Roman
\usepackage[margin=2.5cm]{geometry} % Sets 2.5cm margins
\usepackage{setspace} % Required for double spacing
\usepackage{titlesec}
\usepackage{sectsty}
\usepackage{cite}
\usepackage{graphicx}
\usepackage{soul}
\usepackage{color}
\usepackage{amsmath}
\usepackage{float}
\usepackage{booktabs}
\usepackage{tabularx}
\usepackage{multirow}
\usepackage{enumitem}
\usepackage{algorithm}
\usepackage{algpseudocode}
\usepackage{parskip}
\usepackage[hidelinks]{hyperref}


% --- Settings ---
\doublespacing % Applies double spacing to the document
\sectionfont{\normalsize}
\subsectionfont{\normalsize\it\bfseries}
\subsubsectionfont{\normalsize\it}
\setlength{\parindent}{0pt}
\urlstyle{same}


\begin{document}


% --- Title Page ---
\begin{titlepage}
    \centering
    \includegraphics[width=0.5\textwidth]{Included_Images/PolyU_Logo.png}
    \hfill
    \includegraphics[width=0.25\textwidth]{Included_Images/FENG_Logo.png}
    \vfill
    {\Large 2025/26 Interdepartmental Final Year Project Interim Report\par}
    {\huge \textbf{Intelligent UAV Systems for GNSS-Based Remote Sensing on
    Vegetation} \par}
    \vfill
    {\large [AAE] ZHOU Jiayi (22099961D) \par} {\large [AAE] MAHMUD Md Sahat
    (22097159D) \par} {\large [EEE] GAMAGE Sashenka (22097129D) \par} {\large
    [ME] TAN Qing Lin (22101126D) \par}
    \vfill
    {\large \textbf{Chief Supervisor: } [AAE] Prof. Guohao ZHANG \par} {\large
    \textbf{Co-Supervisor: } [AAE] Prof. Li-Ta HSU \par}
    \vspace{1cm}
    {\large \textbf{Date of Submission: } 2026 January 18 \par}
\end{titlepage}


% --- Table of Contents ---
\pagenumbering{roman}
\tableofcontents
\newpage


% --- Main Content ---
\pagenumbering{arabic}


% --- Abstract ---
\section{Abstract}
Placeholder text


% --- Introduction ---
\section{Introduction}
Vegetation is a critical asset to the environment and the human civilization.
Not only does vegetation produce essential societal resources, but its
distribution and productivity also greatly impacts the terrestrial ecosystems
and the global climate \cite{Richardson2013}. Therefore, the continuous and
accurate monitoring of vegetation is essential for sustainable resource
management \cite{Wallace2006}, ecosystem preservation \cite{Zeng2022}, and
climate change modeling \cite{Richardson2013}. 

Traditionally, methods of vegetation monitoring involved manual field
assessments of site characteristics, extracting vegetation condition indicators
such as species composition, geometrical structure, and biochemical activities
\cite{Gibbons2006}. However, the accuracy and efficiency of such monitoring
methods are highly dependent on the available expertise and resources, as the
site assessments often required the assessor to possess reasonable levels of
field knowledge prior to surveying \cite{Parkes2003}. To address the limitations
of manual assessments and to accommodate for the increasing large-scale
monitoring demands, remote sensing systems emerged as an essential tool for
ecological monitoring \cite{Li2023}. 

Remote sensing refers to the acquisition of an object's information through
measurements obtained without coming into direct contact with said object,
effectively minimizing the need for manual involvement on-site
\cite{campbell2011introduction}. Specifically, remote sensing relies on
information derived from different measurements of energy reflected from the
object of interest \cite{campbell2011introduction}. In the context of surveying
terrestrial vegetation, the remote sensing methods could be classified into two
categories based on the distance between the target object and the measurement
sensor: 1) space-borne and 2) airborne remote sensing. Space-borne remote
sensing involves the use of instruments onboard orbiting satellites, including
spectral and hyperspectral cameras \cite{Qian2022}, synthetic aperture radar
(SAR) \cite{Hu2025}, and space-borne Light Detection And Ranging (LiDAR) sensors
\cite{Bergen2009}. Due to the wide coverage and availability of space-borne
data, large-scale terrestrial changes over time could be captured, enabling the
continuous monitoring of macro-scale ecosystems \cite{Khan2024}. Compared to
space-borne systems, airborne remote sensing can provide significantly improved
spatial resolution and assessment flexibility through the integration of sensors
onboard manned aircraft or unmanned aerial vehicles (UAVs) \cite{Khan2024}.
Although limited by coverage area, airborne remote sensing can provide timely
information for addressing regional emergencies such as pest \cite{Wulder2006}
or wildfire \cite{Arroyo2008} outspreads since the systems could be deployed
on-demand. In UAV-based remote sensing particularly, this temporal flexibility
is further complemented with the benefit of low operational cost, rendering it
an ideal platform for monitoring regional and urban vegetation \cite{Tang2015}. 

Conventionally, UAV remote sensing platforms carry similar instruments to that
of other remote sensing systems, including radar, LiDAR, and multi-spectral or
hyperspectral imagery sensors \cite{Tang2015}. However, while imaging
instruments are susceptible to the influence of lighting and weather
\cite{Wu2024}, LiDAR devices tend to face difficulties penetrating through dense
canopy \cite{Su2007}. Therefore, it is critical to explore a robust sensing
technique that is suitable for UAVs in terms of payload and power consumption to
complement the existing instruments. Recently, the technique of Global
Navigation Satellite System Reflectometry (GNSS-R) is receiving increasing
interest in the field of remote sensing. GNSS-R exploits the L-band signals
transmitted from Global Navigation Satellite Systems (GNSS) that are then
scattered on different terrain surfaces of the Earth \cite{Jin2024}. Then, the
reception of reflected GNSS signals can provide information regarding the
properties of the signal reflector on land \cite{Jin2024}. As signals of
opportunity conventionally dedicated to Positioning, Navigation, and Timing
(PNT) applications, GNSS is capable of providing real-time measurements
regardless of time, location, and weather \cite{Jin2024}. Additionally, as a
bi-static system where signal transmitters are separated from receivers, GNSS-R
is exempt from the need of dedicated transmitter-receiver instruments that are
crucial to mono-static radar systems \cite{Jin2010}. Instead, any consumer-grade
receivers capable of receiving reflected GNSS signals could be used for GNSS-R
remote sensing, further demonstrating GNSS-R's applicability onboard low-cost
remote sensing systems such as an UAV. 

\hl{May need some intro for path planning and lidar here. The above is from
Louise's project proposal, so need further editing. (Also need to mention 
the full name of SLAM somewhere)}

In this project, the technique of GNSS-R will be integrated onto an UAV platform
to achieve an intelligent and structured remote sensing system for vegetation
monitoring. The main objectives of this project are as follows:
\begin{enumerate}
\item To introduce an autonomous path planning framework onboard the UAV platform
for optimized GNSS-R remote sensing.
\item To analyze and model the correlations between signal propagation parameters
retrieved through GNSS-R and ground vegetation conditions.
\item To classify signals reflected by vegetation and predict vegetation parameters
based on raw GNSS data using machine learning-based approach.
\item To establish a detailed 3D canopy map using LiDAR-based SLAM, providing a
spatial validation reference for the 2D vegetation features detected by GNSS-R.
\end{enumerate}

This report is structured as follows: Section 3 reviews contemporary research in
UAV path planning, GNSS-R, machine learning, and LiDAR SLAM, while Section 4
outlines the project's methodologies. The subsequent sections cover the
experiments conducted (Section 5), a discussion of current results (Section 6),
and the project conclusions (Section 7). Finally, Section 8 explores future
work, and Section 9 summarizes project management details. 


% --- Literature Review ---
\section{Literature Review}
To achieve the project objectives, this literature review comprehensively examines
the contemporary research in four domains, each associated with one project objective.
\hl{First, existing path planning methodologies are reviewed to \dots}
Subsequently, the existing techniques of GNSS-R are explored, with a focus on its
prior applications in vegetation parameter retrieval. 
\hl{Then, machine learning \dots}
\hl{Finally, LiDAR-based SLAM techniques \dots}
Consequently, this review identifies the existing gaps in each domain for the 
objective of vegetation monitoring, which will inform the integrated project 
methodologies.

\subsection{Path Planning}
Placeholder text

\subsection{GNSS-R}
The concept of using GNSS-R for remote sensing was first proposed in 1993, where
the correlation between direct and scattered GNSS signals were proven promising
for the application of ocean altimetry \cite{MartnNeira1993APR}. Today, apart
from the originally proposed application, GNSS-R techniques has been extensively
used for Earth observations and land information retrieval, including ocean
salinity, soil moisture, and ice thickness \cite{RodriguezAlvarez2011}. However,
research on vegetation parameter retrieval remains limited to preliminary
analyses \cite{Jia2018}. 

In a previous study, the capabilities of monitoring vegetation through GNSS-R
were demonstrated through a theoretical simulation of GNSS scattering
characteristics \cite{Ferrazzoli2011}. Furthermore, this study revealed that the
sensitivity of GNSS-R signals can be influenced by incidence angle, soil
parameters, and tree size, while signals with lower elevation angles and RL
polarization (transmission of right-hand circular polarization (RHCP) signal,
reception of left-hand circular polarization (LHCP) signal) appeared to be ideal
for forest monitoring \cite{Ferrazzoli2011}. In another study, the polarization
scattering properties of GNSS-R signals in forest canopies were modeled, where
simulations revealed that tree trunk scattering effects would dominate total
scattering response, while satellite azimuth angles are significantly correlated
to the signal polarization \cite{Wu2014}. 

Aside from simulated approaches, empirical or semi-empirical studies regarding GNSS-R
remote sensing of vegetation were also reviewed. In a study by Yueh et al.,
GNSS-R data onboard the Cyclone GNSS (CYGNSS) satellite were analyzed against
the vegetation water content estimated through the satellite-derived Normalized
Difference Vegetation Index (NDVI) \cite{Yueh2022}. The results demonstrated
near-linear relationship between vegetation water content and GNSS-R signal
attenuation, while the CYGNSS data also suggested the existence of volume
scattering within complex forest components \cite{Yueh2022}. Similarly, another
study analyzing GNSS-R data from TechDemoSat-1 demonstrated reduced sensitivity
to soil moisture retrieval due to vegetation attenuation, which could be
effectively compensated using NDVI data, indicating a correlation between signal
attenuation and vegetation water content \cite{Camps2016}. Additionally, the
interference pattern between direct and reflected GNSS signals from the Earth's
surface was used in estimating vegetation height and land topography
\cite{RodriguezAlvarez2011}. Through using a Soil Moisture Interference-pattern
GNSS Observations at L-band (SMIGOL) reflectometer, the instantaneous power of
the direct and reflected signals were coherently added, in which the resulting
power oscillations present notches that are correlated to vegetation layer
thickness and the reflection geometry \cite{RodriguezAlvarez2011}. As a result,
an RMSE of 3-5 cm in estimating the height of vegetation with simple geometrical
structures (barley and wheat) could be achieved \cite{RodriguezAlvarez2011}.
Yet, despite the effort in prior studies, the use of GNSS-R techniques in
vegetation remote sensing and monitoring is far from developed. Most
importantly, existing research relies heavily on space-borne or static
ground-based platforms, while the use of high spatial resolution and flexibility
platforms - such as an UAV - remains underexplored. Therefore, it is critical to
develop a systematic framework for an UAV-based GNSS-R system, in which
correlations between GNSS-R signal features and vegetation condition parameters
are comprehensively assessed and modeled. Additionally, with adequate dataset size,
the GNSS-R framework could be complemented by machine learning-based modelling
approaches, further enhancing the system robustness. \hl{Connection to ML part}

\subsection{Machine Learning}
Placeholder text

\subsection{LiDAR SLAM}
Placeholder text


% --- Methodology ---
\section{Methodology}
Placeholder text

\subsection{UAV Platform}
Placeholder text

\subsection{System Architecture}
Placeholder text

\subsection{Path Planning}
As mentioned in the system architecture, path planning in our system focuses on
two different methods: waypoint-based navigation and coverage path planning.
These methods are crucial in observing various aspects of dense forests, such as
hovering over trees and surveying over a dense canopy to obtain dynamic data for
LiDAR SLAM.


\subsubsection{Waypoint-based navigation}

Waypoint-based navigation can be achieved in various approaches such as
search-based algorithms, graph-based algorithms, and Artificial
Intelligence-based algorithms. Since our system is designed to hover over dense
vegetation environments, identifying potential Fresnel zones, we primarily need
to fly outdoor high altitude open spaces. In order to find the optimal algorithm
to build an intelligent UAV system, we conducted a baseline comparison among the
Djikstra algorithm, the A* algorithm, and the RRT algorithm. 

\begin{table}[ht]
    \centering
    \renewcommand{\arraystretch}{1.3} % Adds a little vertical padding for readability
    % 'l' = left align, 'X' = auto-wrap column
    \begin{tabularx}{\textwidth}{@{} l l X X @{}} 
    \toprule
    \textbf{Category} & \textbf{Algorithm} & \textbf{Advantages} & \textbf{Disadvantages} \\ 
    \midrule

    % SEARCH BASED SECTION
    \multirow{10}{*}{\textbf{Search-based}} & \textbf{A*} & 
    \begin{itemize}[leftmargin=*, noitemsep, topsep=0pt]
        \item Shortest path is guaranteed for a given heuristic map.
        \item Reduces computation power and search space compared to Dijkstra.
    \end{itemize} & 
    \begin{itemize}[leftmargin=*, noitemsep, topsep=0pt]
        \item High computational cost in large or continuous spaces.
        \item Poor scalability in dynamic environments.
    \end{itemize} \\ 
    \cmidrule{2-4}
    
    & \textbf{Dijkstra} & 
    \begin{itemize}[leftmargin=*, noitemsep, topsep=0pt]
        \item Designed to find the optimal path. 
        \item Simple to implement.
    \end{itemize} & 
    \begin{itemize}[leftmargin=*, noitemsep, topsep=0pt]
        \item Computationally expensive.
        \item Impractical for real-time UAV planning.
    \end{itemize} \\ 
    \midrule

    % SAMPLING BASED SECTION
    \textbf{Sampling-based} & \textbf{RRT} & 
    \begin{itemize}[leftmargin=*, noitemsep, topsep=0pt]
        \item Fast exploration of high-dimensional spaces.
        \item Perform well in unknown or partially known maps.
    \end{itemize} & 
    \begin{itemize}[leftmargin=*, noitemsep, topsep=0pt]
        \item Does not guarantee the shortest/optimal path.
        \item Paths are often jagged and require smoothing.
    \end{itemize} \\ 
    \bottomrule
    \end{tabularx}
    \caption{Comparison of Path Planning Algorithms}
    \label{tab:algorithms}
\end{table}

As a result, the comparison shows that the A* algorithm is suitable for offline
path planning, for a given occupancy grid map. Therefore, our system is designed
to perform offline path planning using A* algorithm. On the other hand, RRT
performs well in dynamic unknown environments, which is suitable for online path
planning. 

\paragraph{A* Algorithm Implementation} ~\\
A* algorithm is a heuristic, search based algorithm which calculates the
shortest path from a starting point to an end goal. The algorithm will first
divide the map into nodes, where each node represents a possible position for
the robot to move. These positions depend on the gird architectures such as
4-way connected or 8-way connected grids. In addition to the conventional grid
architecture, novel grid mapping approaches such as NavMeshes are widely used in
3D simulation software. These mapping techniques let the algorithm connects the
locations of the movable agent to the surface (mesh) and stores the surfaces as
convex polygonals. 

In our research, we will be focusing on regular grids which connects nodes
across 26-directions in 3-D space. Each node will be assigned a cost based on the
starting and ending node for a given path. The 2-D path illustration of an A*
algorithm is shown in Figure \ref{fig:astar}.

\begin{figure}[ht]
    \centering
    \includegraphics[width=0.8\linewidth]{Included_Images/A-starmap.jpg}
    \caption{Map Highlighting Shortest Path from Node A to Node K}
    \label{fig:astar}
\end{figure}

The cost function $f(n)$ for each node $n$ is calculated as follows:
\begin{equation}
    f(n) = g(n) + h(n)  
\end{equation}
where:
\begin{itemize}
    \item $g(n)$ is the cost from the starting node to the current node $n$.
    \item $h(n)$ is the estimated heuristic cost from node $n$ to the goal node.
    \item $f(n)$ is the total estimated cost of the shortest path node $n$.
\end{itemize}

In our implementation the cost $g(n)$ (the cost from starting node to end node) is calculated as the Euclidean distance
between the starting node and the current node $n$.

The Euclidean distance can be calculated as follows:
\begin{equation}
    g(n) = \sqrt{(x_n - x_{start})^2 + (y_n - y_{start})^2 + (z_n - z_{start})^2}  
\end{equation}
where $(x_n, y_n, z_n)$ are the coordinates of the current node $n$ and
$(x_{start}, y_{start}, z_{start})$ are the coordinates of the starting node.

The heuristic cost $h(n)$ (the estimated cost from current node $n$ to goal
node) is implemented using the diagonal heuristic function. 

Below is the psuedocode for the diagonal heuristic function:
\begin{center}
    % Change 0.8 to 0.6 or 0.7 to make it even narrower
    \begin{minipage}{0.8\linewidth} 
\begin{algorithm}[H]
\caption{Diagonal Heuristic (3D)}
\label{alg:diagonal_heuristic}
\begin{algorithmic}[1] % The [1] turns on line numbering
    \Require $node_1, node_2$: Current and Goal GridNodes
    \Ensure $h$: Estimated cost to goal
    
    \Function{DiagonalHeuristic}{$node_1, node_2$}
        \State $dx \gets |node_1.x - node_2.x|$
        \State $dy \gets |node_1.y - node_2.y|$
        \State $dz \gets |node_1.z - node_2.z|$
        
        \State \Comment{Find the minimum for 3-D diagonal movement}
        \State $diag_{3d} \gets \min(dx, dy, dz)$
        \State $dx \gets dx - diag_{3d}$
        \State $dy \gets dy - diag_{3d}$
        \State $dz \gets dz - diag_{3d}$
        
        \State $h \gets \sqrt{3} \cdot diag_{3d}$
        
        \State \Comment{Handle remaining 2-D movement}
        \If{$dx = 0$}
            \State $h \gets h + \sqrt{2} \cdot \min(dy, dz) + |dy - dz|$
        \ElsIf{$dy = 0$}
            \State $h \gets h + \sqrt{2} \cdot \min(dx, dz) + |dx - dz|$
        \ElsIf{$dz = 0$}
            \State $h \gets h + \sqrt{2} \cdot \min(dx, dy) + |dx - dy|$
        \Else
            \State \Comment{General case (if dimensions remain)}
            \State $diag_{2d} \gets \min(dx, dy, dz)$
            \State $h \gets h + \sqrt{2} \cdot diag_{2d}$
            \State $remaining \gets (dx - diag_{2d}) + (dy - diag_{2d}) + (dz - diag_{2d})$
            \State $h \gets h + remaining$
        \EndIf
        
        \State \Return $h$
    \EndFunction
\end{algorithmic}
\end{algorithm} 

\end{minipage}
\end{center}

\subsection{GNSS-R}
The GNSS-R framework takes in raw GNSS data collected from the onboard
receivers, extracts raw GNSS measurements, and analyzes the characteristics of
such measurements with respect to the ground features that falls within the
Fresnel zones. Ultimately, the correlation between GNSS measurement features and
vegetation properties would be systematically summarized and modeled. The method
used for retrieving GNSS measurements and the Fresnel zone from collected raw data
is briefly outlined in the following two sections, while the MATLAB code for the 
GNSS-R framework is presented in the Appendix.

\subsubsection{GNSS Measurement Extraction}
Two raw data sources are necessary for the extraction of key GNSS measurements
used in this project: the raw binary GNSS data and the Pixhawk controller log.
First, the binary GNSS data is converted into readable observation files using
RTKCONV \cite{takasu2009rtklib}, and together with the satellite ephemeris data
downloaded from \url{https://cddis.nasa.gov/archive/gnss/data/}, measurements
such as the satellite elevation angle, pseudorange, and carrier to noise ratio
could be extracted using the MatRTKLIB library \cite{taroz:matrtklib}. From the
Pixhawk controller log, the UAV position, which is a filtered positioning result
from various onboard sensors, is obtained and regarded as the ground truth
assuming that the positioning error is small. Ultimately, two major GNSS
measurement are taken for further analysis: carrier to noise ratio ($C/N_0$) and
pseudorange error. 
\begin{figure}[H]
    \centering
    \includegraphics[width=0.8\textwidth]{Included_Images/GNSS_Measurement_Flow.png}
    \caption{Flowchart of extracting GNSS measurements from raw data.}
\end{figure}

\paragraph{Carrier to Noise Ratio} ~\\
Carrier to noise ratio refers to the power of received carrier signal relative
to the noise power per unit bandwidth, usually expressed in decibel-Hertz (dB-Hz)
\cite{joseph2010gnss}. $C/N_0$ can be readily obtained from the receiver, and it 
is calculated using the following equation:
\begin{equation}
    C/N_0 = C - (N - BW)
\end{equation}
where $C$ refers to the carrier power in dBm or dBW; $N$ refers to the noise power
in dBm or dBW; and $BW$ refers to the bandwidth of the observation in Hz.

\paragraph{Pseudorange Error} ~\\
In GNSS, pseudorange refers to the "apparent" distance between a satellite and a
receiver, which includes the true geometric range plus various biases and delays.
For the $i$th satellite, its pseudorange $\rho^i$ could be expressed as the following 
\cite{kaplan2017understanding}:
\begin{equation}
    \rho^i = R^i + \Delta t_r + \Delta t_s^i + I^i + T^i + \epsilon^i
\end{equation}
where $R^i$ refers to the true geometric distance between the satellite and the receiver;
$\Delta t_r$ refers to the receiver clock bias; $\Delta t_s^i$ refers to the satellite
clock bias; $I^i$ refers to the ionospheric delay; $T^i$ refers to the tropospheric 
delay; and $\epsilon^i$ refers to the pseudorange error, which is mainly
introduced through signal reflections. Among all the signals received at a given
instant, a satellite $m$ with the highest elevation angle is selected as the
master satellite, where the assumption that the master satellite is free of reflections
is applied \cite{hsu2018analysis}:
\begin{equation}
    \rho^m = R^m + \Delta t_r + \Delta t_s^m + I^m + T^m.
\end{equation}
Assuming $\Delta t_s^i$ and $\Delta t_s^m$ can be effectively removed by 
satellite-broadcasted parameters, $I^i$ and $I^m$ can be effectively removed by the
Klobuchar model, and $T^i$ and $T^m$ can be effectively removed by the Saastamoinen model,
the pseudorange equations can be approximated as \cite{kaplan2017understanding}:
\begin{equation}
    \rho^i = R^i + \Delta t_r + \epsilon^i,
\label{ith satellite pseudorange}
\end{equation}
\begin{equation}
    \rho^m = R^m + \Delta t_r.
\label{master satellite pseudorange}
\end{equation}
Finally, taking the difference between \eqref{ith satellite pseudorange} and 
\eqref{master satellite pseudorange} and rearranging the equation will result in the
pseudorange error:
\begin{equation}
    \epsilon^i = \rho^i - \rho^m - R^i + R^m.
\end{equation}

\subsubsection{Fresnel Zone Analysis}
Following the extraction of GNSS measurements observed onboard the UAV, the GNSS
signal would be particularly analyzed with respect to the 2-dimensional ground region 
that it is reflected from. Within this region, a specular reflection point could be found,
where the angle of incidence is equal to the angle of reflection. 
\begin{figure}[H]
    \centering
    \includegraphics[width=0.5\textwidth]{Included_Images/Fresnel_Zone_Illustration.png}
    \caption{Schematic diagram of GNSS-R signal scattering geometry.}
\end{figure}
The primary scattering area around the specular point in which reflected signals
arrive at the receiver in coherence is called the first Fresnel zone (FFZ). By
definition, it is an ellipse for which the signal phase change across is within
$\frac{\lambda}{2}$ radians, where $\lambda$ refers to the signal wavelength
\cite{Jia2018}. Given the incidence angle $\theta$ and the height of the
receiver $H$, the semi-major axis $a$ and the semi-minor axis $b$ of the FFZ can
be calculated as the following \cite{yu2021theory}: 
\begin{equation}
    a = \frac{\sqrt{\lambda H sin(\theta)+\frac{\lambda^2}{4}}}{(sin(\theta))^2},
\end{equation}
\begin{equation}
    b = \frac{\sqrt{\lambda H sin(\theta)+\frac{\lambda^2}{4}}}{sin(\theta)}.
\end{equation}
To calculate the location of the FFZ center, a local coordinate system is
defined. The origin is the receiver's ground projection, while the y-axis aligns
with the ground projection of the receiver-to-satellite vector. The z-axis
follows the local East-North-Up (ENU) "up" direction, and the x-axis completes
the right-handed system by remaining orthogonal to both x and y axes. 
\begin{figure}[H]
    \centering
    \includegraphics[width=0.6\textwidth]{Included_Images/Fresnel_Zone_Coordinates.png}
    \caption{Schematic diagram of the local coordinate system defined.}
\end{figure}
In this coordinate system, the location of the FFZ center could be expressed as
$(x_c, y_c, z_c)$, where \cite{yu2021theory}:
\begin{equation}
    x_c = 0
\end{equation}
\begin{equation}
    y_c = (\frac{\lambda}{2}+H sin(\theta))\frac{cos(\theta)}{(sin(\theta))^2}
\end{equation}
\begin{equation}
    z_c = 0
\end{equation}
The flowchart for calculating the FFZ shape and location is outlined in the
figure below. Here, since the UAV true position is given in the global
coordinate of latitude, longitude, and height (LLH) in the Pixhawk log data, the
terrain height shall be subtracted to obtain the UAV height above ground, $H$.
This terrain height is obtained through the Hong Kong Digital Terrain Model
(\url{https://data.gov.hk/en-data/dataset/hk-landsd-openmap-5m-grid-dtm}), which
shows the topographical information in 5 meter by 5 meter grids with an accuracy of
$\pm$ 5 meters \cite{dataDigitalTerrain}. 
\begin{figure}[H]
    \centering
    \includegraphics[width=0.75\textwidth]{Included_Images/FFZ_Flow.png}
    \caption{Flowchart of calculating FFZ shape and location from collected data.}
\end{figure}


\subsection{Machine Learning}
Placeholder text

\subsection{LiDAR SLAM}
\subsubsection{Frontend: LiDAR-Inertial Odometry}
For the front end of the modular SLAM system, LiDAR-Inertial Odometry (Fast-LIO)
is used to build a map effectively within a short amount of time. To do this,
Iterative Extended Kalman Filter (IEKF) is used to fuse the data between IMU and LiDAR
point clouds to create an odometry of the environment. 

Iterative Extended Kalman Filter (IEKF) follows the core of Bayesian Recursion
to estimate the state of the robot through the prediction and measurement update
step. In the prediction step, IEKF assumes that the state of the system at time
$k$ evolved from the prior state at time $k-1$ is shown as follows:
\begin{equation}
x_k = f(x_{k-1}) + w_{k-1}
\end{equation}

Where $x_k$ is the state vector containing the terms of interest for the system
at time k. The $f(.)$ represents a non-linear state function that is used to
forecast current state data from prior state data. We can approximate $w_k-1$ as
$N(0,Q_k)$ where it has a zero-mean Gaussian distribution with covariance matrix
$Q_k$.

The IEKF is initialized with 


% --- Experiments ---
\section{Experiments}
Placeholder text

\subsection{Experiment Workflow}
Placeholder text

\subsection{Summary of Experiments Conducted}
Placeholder text

\subsection{Key Experiment 1}
Placeholder text

\subsection{Key Experiment 2}
Placeholder text


% --- Results and Discussion ---
\section{Results and Discussion}
Placeholder text

\subsection{Path Planning}
Placeholder text

\subsection{GNSS-R}
Placeholder text

\subsection{Machine Learning}
Placeholder text

\subsection{LiDAR SLAM}
Placeholder text


% --- Conclusion ---
\section{Conclusion}
Placeholder text


% --- Future Works ---
\section{Future Works}
Placeholder text

\subsection{Path Planning}
Placeholder text

\subsection{GNSS-R}
Placeholder text

\subsection{Machine Learning}
Placeholder text

\subsection{LiDAR SLAM}
Placeholder text


% --- Project Management ---
\section{Project Management}
Placeholder text

\subsection{Gantt Chart}
Placeholder text

\subsection{Project Difficulties and Solutions}
Placeholder text

\subsubsection{Path Planning}
Placeholder text

\subsubsection{GNSS-R}
Placeholder text

\subsubsection{Machine Learning}
Placeholder text

\subsubsection{LiDAR SLAM}
Placeholder text


% --- Appendix ---
\newpage
\addcontentsline{toc}{section}{Appendix}
\section*{Appendix}

\subsubsection*{GitHub Repositories}
The code scripts of this project are organized in various GitHub repositories, which
aligns with different frameworks of this project. Specifically: 
\begin{itemize}
    \item GNSS-R: \url{https://github.com/zhlouise/UAV_GNSS-R}
\end{itemize}


% --- References ---
\newpage
\addcontentsline{toc}{section}{References}
\bibliographystyle{IEEEtran}
\bibliography{References.bib} 


\end{document}